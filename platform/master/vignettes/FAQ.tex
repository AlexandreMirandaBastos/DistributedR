
%%%%%%%%%%%%%%%%%%%%%%%%%%%%%%%%%%%%%%%%%%%%%
% FAQs
% Indrajit Roy (Nov, 2013)
%%%%%%%%%%%%%%%%%%%%%%%%%%%%%%%%%%%%%%%%%%%%%

\documentclass[10pt, twocolumn]{article}
\usepackage{enumitem}
\usepackage{balance}

\begin{document}

\title{Distributed R Frequently Asked Questions}
\author{HP Vertica Development Team}
\date{}
\maketitle

%\twocolumn[
%\begin{@twocolumnfalse}
%\abstract{This document provides answers to commonly asked questions about Distributed R.}
%\end{@twocolumnfalse}
%]

\section{Overview}

\begin{description}[style=nextline]
\item [What is Distributed R?]  Distributed R is a High-Performance
  Scalable Platform for the R language. It enables R to leverage
  multiple cores and multiple servers to perform advanced Big Data
  analytics. It consists of new R language constructs to easily
  parallelize algorithms across multiple R processes.


\item [When should I use Distributed R?]  Distributed R allows you to
  overcome the scalability and performance limitations of R to analyze
  very large data sets. Distributed R provides the ability to run
  analysis on the complete data set and not just the sample.

\item [Which algorithms are implemented using Distributed R?]
  Generalized Linear Models (glm) for performing many forms of
  regressions - includes logistic, linear, and poisson regression. For
  more details refer to HPDGLM-Manual.pdf and HPDGLM-UserGuide.pdf.

\item [How can I convert existing R programs to scale using
  Distributed R?]  Distributed R provides simple and powerful tools
  for distributed computing in R. Data-structures such as distributed
  array, {\em darray}, enables R algorithms to handle big data by
  distributing data across machines in a cluster.  Language primitives
  such as {\em foreach} in Distributed R enables R developers to
  easily parallelize algorithms.  To convert an R program to scale
  using Distributed R, the programmer should use data-parallelism
  techniques to break the program into smaller sequential functions,
  apply them on data partitions, and then aggregate partial results.
  Please refer to Distributed-R-UserGuide.pdf to understand the
  programming model

\item [Can I use other R packages with Distributed R?]
  Yes. Distributed R packages can be loaded with other R packages in
  the Comprehensive R Archive Network (CRAN). Distributed R can
  execute parallel programs that call existing R packages. For more
  details refer to the ‘Parallel execution using existing packages’
  section in Distributed R-UserGuide.pdf.

\item [Can I benefit from running Distributed R in a single multi-core
  machine?  ]  Yes. Distributed R can leverage multiple cores by
  distributing processing across multiple cores similar to multi-node
  cluster.

\item [How do I use algorithms developed on Distributed R?]  Usage of
  Distributed R algorithm functions are similar to any other R based
  algorithm function. However, Distributed R algorithms use
  distributed datastructures instead of simple arrays. For example,
  hpdglm implements a distributed alternative for R glm. The signature
  of hpdglm is the same as R glm, except that it uses distributed
  arrays as input instead of simple arrays. For more details refer to
  ‘HPDGLM-UserGuide.pdf’

\item [What are the hardware requirements to run Distributed R?]  You
  can run Distributed R on commodity hardware. You need enough total
  memory to hold all of the data you want to analyze, along with a
  buffer for R's bookkeeping.  If you have hardware with more memory,
  you will need fewer total nodes to do the analysis.  We recommend a
  stand-alone cluster of HP DL380’s.

\item [Can I run Distributed R in the cloud or on an appliance?]
  Distributed R is not currently tested in the cloud or on HP Vertica
  appliances.

\item [Is Distributed R part of HP Vertica?  Can I use any database
  with Distributed R?]  Distributed R is a collaboration project
  between HP Labs and HP Vertica.  It was created at HP Labs.  Today,
  we are working together to make an offering of it.  You can store
  your data in HP Vertica and use our optimized connector vRODBC to
  load data into Distributed R. You can load data from other sources
  as well.

\item [Does Distributed R replace HP Vertica R UDX?]  No.  Distributed
  R does not replace HP Vertica R UDx.  R UDx works best when your
  data can fit into memory on a single node and you are satisfied with
  the processing time.  Distributed R is a way for us to improve our
  offering in this space.

\item [Does Distributed R work with RStudio?]  Distributed R works
  with common R development environments. However, we have primarily
  tested it with the default R console.

\end{description}

\section{Installation}

\begin{description}[style=nextline]
\item [What are the steps to install Distributed R?]  Prior to installing
  Distributed R ensure you have a supported Linux operating system and
  software pre-requisites installed. The installation document,
  Distributed-R-Installation-Guide.pdf, describes detailed
  installation steps.

\item [Can I install Distributed R on a single server?]  Yes. If you
  have a single server with multiple cores, Distributed R enables you
  to leverage multiple-cores to improve performance. The installation
  document section ‘Running Distributed R – Single Machine mode’
  provides more details of how to install and use Distributed R in a
  single machine.

\item [How do I verify the successful installation of Distributed R in
  a cluster environment?]  Use {\tt distributedR\_status()} command to
  verify that all nodes up and running in a cluster.
\end{description}

\section{Performance and scalability}

\begin{description}[style=nextline]
\item [What is the upper limit of the size of data that Distributed R
  can analyze?]  We have tested regression on more than 1 terabyte of
  data.  We tested few permutations of this data set size, varying the
  number of rows and columns.  We would like to understand your
  scalability and data size needs. Please either post your
  requirements to our beta email distribution list,
  {\em VerticaDistributedRBeta@external.groups.hp.com}, or email us at {\em
    sunil.venkayala@hp.com} and {\em geeta.aggarwal@hp.com}.

\item [How do you characterize the scalability of Distributed R?]  We
  have noticed near linear scaling with regression.  This behavior is
  with a fixed data set and increasing number of nodes.  We have also
  observed weak scaling (execution time remains nearly constant when
  number of nodes and data set is increased proportionally). We are
  working on characterizing scalability based on the type of algorithm
  you are using.

\item [I have a small dataset. Should I use Distributed R?] It is
  known that distributed processing systems incur additional overheads
  due to communication and bookkeeping. Therefore, Distributed R may
  not be beneficial to you if your dataset size is small and the execution
  time of your program is in seconds or a minute. Use vanilla R in
  such cases.

\item [How do I load data from HP Vertica to Distributed R?] Use
  connectors such as RODBC to load data from HP Vertica. We provide a
  fast ODBC connector called vRODBC. Additionally, the HPDGLM package has a default
  {\tt dataLoader} function that uses ODBC connections to load data in
  parallel from HP Vertica.

\item [What happens if a node goes down?]  Distributed R detects that
  the node is down and prompts you to restart the session.

\item [What happens if I load Distributed R with too much data?]
  Distributed R prompts you that there is not enough memory and asks
  you to restart.

\item [How much hardware do I need for a particular data set size?]
  The rule of thumb is that you need enough servers so that the whole
  dataset fits into the total memory of the machines, along with a
  buffer for bookkeeping by R. If you want the computations to run faster, you
  need to use more CPU cores.

\end{description}
\section{Limitations}

\begin{description}[style=nextline]
\item [What are the current limitations of Distributed R?] Distributed
  R is in beta stage and we want to use this opportunity to obtain
  feedback. We are actively improving the system. Some limitations are listed below:
\begin{itemize}
\item Distributed R currently supports only one session on a
  cluster. Multiple users may not be able to run concurrently on the
  same machines.
\item Debugging distributed programs is, in general, challenging. We
  plan to add support for debugging tools for Distributed R. If you
  make a programming mistake or have a bug in the code, you may need
  to restart the Distributed R session.
\end{itemize}

\balance
 
\end{description}
\end{document}
